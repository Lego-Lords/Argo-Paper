%%%%%%%%%%%%%%%%%%%%%%%%%%%%%%%%%%%%%%%%%%%%%%%%%%%%%%%%%%%%%%%%%%%%%%%%%%%%%%%%%%%%%%%%%%%%%%%%%%%%%%
%
%   Filename    : chapter_3.tex 
%
%   Description : This file will contain your Research Methodology.
%                 
%%%%%%%%%%%%%%%%%%%%%%%%%%%%%%%%%%%%%%%%%%%%%%%%%%%%%%%%%%%%%%%%%%%%%%%%%%%%%%%%%%%%%%%%%%%%%%%%%%%%%%

\chapter{Research Methodology}
This chapter lists and discusses the specific steps and activities that will be performed by the proponents to accomplish the project. The discussion covers the activities from pre-proposal to Final Thesis Writing.

\section{Review of Related Works and Planning}
Research works involving Assembly Assisted by Augmented Reality will continue to be reviewed by the proponents in order to be more knowledgeable on the essential background for the study. It is also necessary to review the recent works and advancements in augmented reality technology in order to discover which functionalities, algorithms may be included in the application itself. From this, the requirements for the system will be decided upon. The evaluation methods and metrics to be used for checking whether the system has met its goals will also be decided upon in this stage based on research works that have checked the effectiveness of augmented reality assembly guidance against other methods.

\section{Data Gathering}
Different LEGO models will be selected for this stage, specifically the most commonly used models for LEGO modelling. The 3D vowel models of the stages used in assembling the LEGO models will be acquired. This will serve as the input for the system.

\section{Prototype Development}
This phase, the development of the working prototype includes the formulation of the system design, implementation and the testing and analysis of the system. This series of activities will be performed iteratively until all the requirements decided upon during the planning phase are met.

\subsection{Design}
The system design will be based on the requirements that need to be fulfilled as well as the knowledge gained from the reviewed works with a similar context. The design will include designing the overall software architecture as well as the user interface of the system and the flow of how the system would work.

\subsection{Implementation}
After designing the system, the next activity would be the implementation of the system itself. The programming of the system's components will be based on the algorithms that were reviewed earlier as well as the decided flow and design. 

\subsection{Testing and Evaluation}
Testing will be performed to ensure that the implementation is correct and working as designed. After which, the prototype will be evaluated using the evaluation methods and metrics that were planned on earlier. The results from this as well as the issues and problems encountered will be analyzed.

\section{Documentation}
Documentation will be performed throughout the entire duration of the project from the activities during the pre-proposal until the Final Thesis writing. This is in order to keep track of the research progress during each stage. Issues and problems that will be experienced and how they will be resolved if applicable will also be documented.

\begin{comment}
\section{Research Activities}
Research activities include inquiry, survey, research, brainstorming, canvassing, consultation, review, interview, observe, experiment, design, 
test, document, etc.  The methodology also includes the following information:

\begin{itemize}
   \item who is responsible for the task
   \item the resource person to be contacted
   \item what will be done
   \item when and how long will the activity be done
   \item where will it be done
   \item why should be activity be done
\end{itemize}
\end{comment}
\begin{landscape}
\section{Calendar of Activities}

Table \ref{tab:timetableactivities} shows a Gantt chart of the activities.  Each bullet represents approximately
one week worth of activity.

%
%  the following commands will be used for filling up the bullets in the Gantt chart
%
\newcommand{\weekone}{\textbullet}
\newcommand{\weektwo}{\textbullet \textbullet}
\newcommand{\weekthree}{\textbullet \textbullet \textbullet}
\newcommand{\weekfour}{\textbullet \textbullet \textbullet \textbullet}

%
%  alternative to bullet is a star 
%
\begin{comment}
   \newcommand{\weekone}{$\star$}
   \newcommand{\weektwo}{$\star \star$}
   \newcommand{\weekthree}{$\star \star \star$}
   \newcommand{\weekfour}{$\star \star \star \star$ }
\end{comment}


\begin{table}[ht]   %t means place on top, replace with b if you want to place at the bottom
\centering
\caption{Timetable of Activities} \vspace{0.25em}
\begin{tabular}{|p{2in}|c|c|c|c|c|c|c|c|c|c|c|c|} \hline
\centering Activities (2016-17) & Aug & Sep & Oct & Nov & Dec & Jan & Feb & Mar & Apr & May & Jun & Jul \\ \hline
Review of Related Works and Planning  & \weekfour & \weekfour & \weekfour & \weekfour &  &  &  &  &  &  &  &  \\ \hline
Data Gathering &  &  &  & \weekfour & \weektwo~~~ &  &  &  &  &  &  &  \\ \hline
Design  &  &  &  &  &  & ~~~\weekthree & \weekfour &  &  &  &  &  \\ \hline
Implementation &  &  &  &  &  &  &  & \weekfour & \weekfour & \weekfour &  &  \\ \hline
Testing and Analysis &  &  &  &  &  &  &  &  &  &  & \weekfour & \weektwo~~~ \\ \hline
Documentation & \weekfour  & \weekfour & \weekfour & \weekfour & \weektwo~~~ & ~~~\weekthree & \weekfour & \weekfour & \weekfour & \weekfour & \weekfour & \weektwo~~~ \\ \hline
\end{tabular}
\label{tab:timetableactivities}
\end{table}
\end{landscape}
